\begin{frame}[containsverbatim]
\frametitle{Beamer-Themes für das TUC2014-Layout}

\begin{itemize}
\item Für das TUC-Layout wurde ein eigenes Beamer-Theme erstellt.
\item Es kann über \lstinline[language={[LaTeX]TeX}]+\usetheme{tuc2014}+ geladen werden.
\item Dadurch werden wiederum folgende Unter-Themes geladen:
  \begin{description}
  \item[Color Theme] Legt das Farbschema für jeder Fakultät fest.
  \item[Font Theme]  Lädt die Hausschrift "`Roboto Condensed"'.
  \item[Inner Theme] Legt dir Formatierung von Hervorhebungen, Aufzählungslisten,
                     Titelseite und Inhaltsverzeichnis fest.
  \item[Outer Theme] Stellt die Formatierung von Kopf- und Fußzeile sowie der
                     linken Logospalte einer Folie ein.
  \end{description}

\bigskip

\item Die Teil-Themes können ggf. auch unabhängig voneinander genutzt werden.
  \begin{itemize}
  \item In den meisten Fällen wird es aber nur Sinn machen das Color Theme
        und/oder das Font Theme im Kombination mit anderen Beamer-Themes zu
        nutzen.
  \end{itemize}
\end{itemize}
\end{frame}


\begin{frame}[containsverbatim]
\frametitle{Zusatzfunktionen des Haupt-Themes}

\begin{itemize}
\item Folgende Einstellungen werden Haupt-Theme \structure{zusätzlich} zum
      Laden der Unter-Themes vorgenommen.
  \begin{itemize}
  \item \lstinline[language={[LaTeX]TeX}]+\usefonttheme{professionalfonts}+ \\
        für die üblichen Mathematik-Schriften
  \smallskip
  \item \lstinline[language={[LaTeX]TeX}]+\setbeamercovered{transparent}+ \\
        für die Schattierung ausgeblendeter Folieninhalte
  \smallskip
  \item Ferner werden die Folien im Zweibildschirmmodus (siehe Kapitel~19
        in \cite{beamerguide}) nur klein skaliert, wenn tatsächlich Notizen
        für diese Folie hinterlegt wurden.
  \end{itemize}
\end{itemize}
\end{frame}
