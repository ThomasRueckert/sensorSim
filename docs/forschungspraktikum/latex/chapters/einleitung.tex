\chapter{Einleitung}

\paragraph{Allgemeines}
Es kann erforderlich sein, dass viele verschiedene Messwerte eines Gebietes über längere Zeiträume hinweg ermitteln werden müssen. Damit man viele, regelmäßige und genaue Messdaten erhält, können Sensorknoten eingesetzt werden, welche in großen Sensornetzwerken agieren können. Diese können sehr viele und verschiedene Knoten enthalten.\\
Ein kritischer in Sensornetzwerken ist die Lebensdauer. Daher sollte der Energieverbrauch so weit wie möglich optimiert werden, damit trotz kleiner, günstiger Knoten eine ausreichend lange Arbeitszeit garantiert werden kann. Dazu kann Hardware mit niedrigem Stromverbrauch genutzt werden. Außerdem kann ein Wake-up-Receiver eingesetzt werden. Dadurch können Sensorknoten ihren Funktransreceiver abschalten, solange sie keine Kommunikation mit anderen Knoten ausführen müssen. Sobald eine Nachricht zum Wecken der Knoten vom Wake-up-Receiver empfangen wurde, wird der Funktransreceiver wieder zugeschaltet und es kann die normale Kommunikation stattfinden. Es können spezielle Knoten gewählt werden, die das Verwenden dieser Nachrichten ausführen und somit den Wechsel zwischen den verschiedenen Phasen steuern. Sie können zusätzlich durch Nachrichten zum Beispiel Messungen der anderen Knoten steuern.\\
Eine mögliche Implementierung zur Strukturierung ist eine Unterteilung der Knoten in Cluster mit je einem Masterknoten. Zusätzlich kann eine Datensenke zur Verarbeitung und Speicherung der Daten hilfreich sein können, welche als Schnittstelle nach außen fungiert. Mit verschiedenen Routingverfahren kann die Kommunikation in den Netzwerken gesteuert werden. Durch verschiedene Maßnahmen wie diese kann dies auch beim Einsparen von Energie helfen.\\
\\
\paragraph{Implementierung}
Es sollen verschiedene Szenarios erstellt werden, welche Netzwerke mit unterschiedlichen Konfigurationen und Routingverfahren implementieren. In den Netzwerken sollen Sensorknoten mit verschiedenen Arten von Sensoren eingesetzt werden. Die Knoten sollen das Funkmodul 802.14.4 / CC2420 nutzen. Zusätzlich sollen sie einen Wake-up-Receiver besitzen. Weiterhin soll sich im Schlafzustand der Stromverbrauch ändern und das Empfangen von Paketen deaktiviert sein. Das Netzwerk soll sich in einer Umgebung befinden, welche verschiedene Messdaten für die Sensoren bereitstellt.\\
Es soll zum einen ein Szenario mit einem simplen Routingverfahren implementiert werden, zum Beispiel anhand einer fest definierten Baumstruktur. Ein zweites Szenario soll das ApplicationClustering benutzen.\\
Dabei sollen Messdaten wie zum Beispiel der Energieverbrauch, der Ladezustand über die Zeit, die gesamte Netzwerklebenszeit und Informationen über ausgefallene Knoten erhoben werden.