\def\abstractname{Abstract} 	% Wenn der Text "Zusammenfassung" erscheinen soll, dann muß dies auskommentiert werden
				
\begin{abstract}
%%
%% Inhalt der Arbeit
%%
In dieser Arbeit soll die Simulation von Sensorknoten in einem Netzwerk untersucht werden. Dabei sollte zum einen die Kommunikation zwischen den verschiedenen Knoten, von denen es viele und verschiedene geben soll betrachtet. Diese führen eine kabellose Kommunikation miteinander. 
Die einzelnen Konten besitzen Sensoren, die verschiedene Umweltparameter auslesen. Die Bereitstellung dieser Umweltparameter in der Simulationsumgebung ist auch ein Teil der Implementierung. Außerdem sind die Knoten batteriebetrieben und auch deren Energieverwaltung wurde betrachtet. Nach Test und Modellierung ist die Auswertung und Visualisierung von Simulationsdaten von entscheidender Rolle. 
Als Simulationsumgebung und -sprache wurde Omnet++ mit dem Framework MiXiM, welches Grundfunktionen für mobile Knoten mit kabelloser Kommunikation bereit stellt.
\end{abstract}