%======================================================================
%	Vorlage
%======================================================================
%	$Id$
%	Matthias Kupfer
%======================================================================
%	Documentclass
%======================================================================
\documentclass[
%	draft,			% Entwurfsmodus: Bilder als Rahmen, Überlängen werden deutlich markiert
%	10pt,			% 
%	11pt,			% KOMA default
	12pt,			% 
	a4paper,		% DIN A4
	twoside,		% Zweiseitig
	german,			%
%----------------------------------------------------------------------
% Die folgenden Befehle stammen aus dem KOMA Paket
	headsepline,		% Linie unter der Kopfzeile	
%	headnosepline,		%
%	foodsepline,		% Linie über Fussnote	
	footnosepline=true,		%
	automark,		% Kolumnentitel lebendig
%	bigheadings,		% default: Überschriften gross setzen
%	headings=normal,		% Überschriften normal setzen
	smallheadings,		% Überschriften eher klein setzen
%	pointlessnumbers,	% Keinen Punkt hinter die letzte Zahl eines Kapitels (auch bei Anhang)
%	chapterprefix,		% Kapiteluberschriften "Kapitel"
	appendixprefix,		% Anhang
	openright,		% Kapitel auf der rechten (ungeraden) Seite anfangen lassen
%	openany,		% 
%	cleardoublestandard,	%
%	cleardoublepage=plain,	%
	cleardoubleempty,	% echt leere seiten zwischen kapiteln, keine seitenzahl
	abstracton,		%
	index=totoc,		% Index soll im Inhaltsverzeichnis auftauchen
	listof=totoc,		%
	bibliography=totoc,		%
%	parskip, 		% parskip-, parskip*, parskip+
% 	halfparskip, 		% halfparskip-, halfparskip* und halfparskip+
%	DIVclassic,
 	BCOR8mm,		%%?? Bindungskorrektur: 
				% BCOR<Breite des Bindungsverlustes> 
]{scrreprt} 

\setlength{\parindent}{0pt} % Absatzformatierung (Einrückung und Zeilenabstand)
\setlength{\parskip}{2ex} 

%======================================================================
%	Bearbeitung sowohl mit LaTeX als auch mit pdfLaTeX ermoeglichen
%======================================================================
%\newif\ifpdf 
	%\ifx\pdfoutput\undefined 
	%\pdffalse 			% we are not running pdflatex 
%\else 
	%\pdfoutput=1 			% we are running pdflatex 
	%\pdftrue 
%\fi


%======================================================================
%	Verwendete Pakete
%======================================================================

%\usepackage{babel}		% Sprachen

%% Latex mit deutschen Umlauten:
%% http://www.cs.albany.edu/~herrmann/latex_umlaute/

% \usepackage[latin1]{inputenc}   % Eingabe von ü,ö,ä,ß erlaubt 
\usepackage[utf8]{inputenc}


\usepackage[T1]{fontenc}	% EC-Schriften verwenden (vs. DC) da 8-Bit
				% EC-Schriften als T1-kodierten CM-Schriften
				% European/Ext.-Computer-Modern-(EC)-Schriften
				% Umlaute, Anführungszeichen ...
				% => Umlauten koennen richtig getrennt werden
				% FAQ 5.3.2
								
\usepackage{ae,aecompl}		% virtuelle-CM-Fonts
				% da EC nicht als PostScript-(Type-1) verfuegbar
				% => keine echten Umlaute im Dokument
				%(Problem bei Suche)
				% By loading the ae package (\usepackage{ae}), 
				% you loose some characters as mentioned in 
				% README. 
				% The package aecompl by Denis Roegel restores
				% these characters which are taken from the ec 
				% fonts. If you use pdftex, you will get these 
				% characters as bitmaps, but this might be 
				% better than not having them at all.

%\usepackage{times, mathptm}	% TimesNewRoman Schrift (Acrobat Reader Fonts), 
				% dazu braucht man auch den entsprechenden 
				% Zeichensatz für den Math-Mode
%\usepackage{pslatex}		% ? mathematische Formeln mit Standard 
				% Postscript Fonts gesetzt
			% Paket     Roman         Serifenlos  Typewriter
%\usepackage{times}	% -----------------------------------------------
			% times     Times         Helvetica   Courier
			% palatino  Palatino      Helvetica   Courier
			% newcent   NewCenturySch AvantGarde  Courier
			% bookman   Bookman       AvantGarde  Courier
			% Diese Schriften sind die Standard-PostScript-Schriften
			% und in jedem Drucker verfügbar

\usepackage{
%	german,			% Deutsche Trennungen (ALTE Rechtschreibung), 
				% Anführungsstriche und mehr 
	ngerman,		% Deutsche Trennungen (NEUE Rechtschreibung), 
				% Anführungsstriche und mehr
				%
%	acronym,		% Verwaltung von Abkuerzungen
	bibgerm,		% Deutsche Bibliographie; notwendig für Bibtex 
	calc,			% Erweiterung der arithmetischen Funktionen in 
				% LaTeX
				% wird verwendet um Titelseite zu zentrieren
	color,			% im Laufenden Text einfach mit \color{Farbe) zwischen den 
				% Farben umschalten, wobei Farbe einfach 
				% durch z.B. red, blue, black etc. ersetzt wird
				% \textcolor{farbe){Text)
%	epigraph,		% Zitat am Kapitelanfang
%	fancyhdr,		% Kopf- und Fußzeilen von Dokumenten frei 
				% gestalten
	fancybox,		% shadowbox, doublebox, ovalbox, Ovalbox 
	fancyvrb,		% verbatim Erweiterung:
	float,			% Positionierung von Gleitobjekten genau an der Stelle, wo man
				% 'figure'- oder 'table'-Umgebung die 
				% Positionierung [H] gesetzt werden
%	glosstex,		% Glossar und Abkürzungsverzeichnis
	mdwlist,		% compact list: itemize* ..
	scrdate,		% \todaysname 
	scrtime,		% \thistime
	scrpage2,		% Kopf- und Fußzeilen flexibel gestalten
				%
%	moreverb,		% verbatim-ähnlich: boxedverbatim, listing
%	verbatim,		% Darstellung von "Text, wie er eingegeben wird"
				%
%	lscape,			% Erstellt eine um 90% gedrehte *neue* Seite
%	textcomp,		% Sonderzeichen
	booktabs,		% Tabellenlinien
%	longtable,		% Tabellen > 1 Seite
%	supertabular,		% Tabellen > 1 Seite
	tabularx,		% Blocksatzspalten
    multirow,
%	ltxtable,		% tabularx + longtable
	multicol,		% mehrspaltige Zeilen
%	varioref,		% einheitliche Verweise
%	endnotes,		% Fussnoten -> Endnoten
%	rotating,		% sidewaystable und sidewaysfigure
%	natbib,			% Bibliographie ohne Klammer etc.
%	marvosym,		% Euro etc.
}

%\usepackage{pstricks}
\usepackage{listings}



%\usepackage{wasysym}

%\usepackage[german, first, bottomafter]{draftcopy}

%\usepackage{setspace}	% Durchschuß, Zeilenabstand
%\doublespace		% doppelzeilig oder
%\onehalfspacing	% anderthalbzeilig

% Für schöne Darstellung von Algorithmen
%\usepackage[german, algoruled, algochapter]{algorithm2e}
%\usepackage{algorithmic}
%\usepackage[chapter]{algorithm}
%\floatname{algorithm}{Algorithmus}

\usepackage{url}      		% irgendwas mit urls
\usepackage[table]{xcolor} % farbige Zeilen in Tabellen

%======================================================================
%	Bilder, Links
%======================================================================

\usepackage{graphicx}
\graphicspath{{img/}}	% Angabe der Pfade, wo die Grafiken liegen; mehrere Pfade sind möglich

\usepackage{thumbpdf}
\usepackage[
   colorlinks,        % Links ohne Umrandungen in zu wählender Farbe
   linkcolor=black,   % Farbe interner Verweise
   filecolor=black,   % Farbe externer Verweise
   citecolor=black,   % Farbe von Zitaten
   urlcolor=black
]{hyperref}

%======================================================================
%	Einstellungen
%======================================================================

%\typearea{11}		% Satzspiegel neu konstruieren (KOMA)
			%      10pt 11pt 12pt 
			% DIV : 8   10   12  

\pagestyle{scrheadings}	% Standart  Kopf- und Fußzeile
\setkomafont{pageheadfoot}{\small\scshape}

% ---------------------------------------------------------------------
%\setcounter{secnumdepth}{2}
%\setcounter{chapter}{-1}
\setcounter{tocdepth}{3}

% ---------------------------------------------------------------------
%\sloppy		% weniger Worttrennungen, größere Wortabstände
\fussy			% viele Worttrennungen, "schönere" Wortabstände

% ---------------------------------------------------------------------
%\flushbottom			% Ausrichtung der Seitenenden jeweils auf gleicher Höhe
\raggedbottom

% ---------------------------------------------------------------------
%\sloppypar		% Das hier relaxt die Einstellungen zum Wortabstand 
			% extrem. Damit ragen keine Worte über den rechten 
			% Zeilenabstand hinaus. Dafür muß stärker auf
			% Wortabstand geachtet werden, der kann dann ziemlich 
			% groß werden. Man erhält aber keine Meldung mehr 
			% über underfull boxes.

%Hiermit kann man das gleiche mit weniger Holzhammer erreichen:
%\setlength{\tolerance}{2000}           % Strafpunkt für Zeilenumbruch
%\setlength{\emergencystretch}{3pt}     % Soweit dürfen einzelne Worte mehr 
					% auseinandergezogen werden
%\setlength{\hfuzz}{1pt}                % Macht den rechten Rand um bis zu 1pt 
					% flatterig.

% ---------------------------------------------------------------------
%Vermeiden einzelner Zeilen am Ende einer Seite oder oben auf einer neuen Seite
\clubpenalty10000
\widowpenalty10000

% ----------------------------------------------------------------------
% neue Umgebungen für verwendete Sätze und Beispiele
\newtheorem{bsp}{Beispiel}[chapter]
\newtheorem{satz}{Satz}[chapter]

\lstset{
  language        = php,
  basicstyle      = \small\ttfamily,
  keywordstyle    = \color{blue},
  stringstyle     = \color{red},
  identifierstyle = \color{black},
  commentstyle    = \color{gray},
  showstringspaces=false,
  emph            =[1]{php},
  emphstyle       =[1]\color{black},
  emph            =[2]{if,and,or,else},
  emphstyle       =[2]\color{yellow},
  emph            =[3]{use, class, function, public, extends, \$this},
  emphstyle       =[3]\color{blue},
  emph            =[4]{\$modules},
  emphstyle       =[4]\color{blue}
}
\lstdefinelanguage{ned}
{
}
\lstset{
  language        = ned,
  basicstyle      = \small\ttfamily,
  keywordstyle    = \color{blue},
  stringstyle     = \color{red},
  identifierstyle = \color{black},
  commentstyle    = \color{gray},
  showstringspaces=false,
  emph            =[1]{simple, module, network, extends, channel, channelinterface,import,like,moduleinterface,package,property},
  emphstyle       =[1]\color{blue},
  emph            =[2]{if,and,or,else,for,input,output,inout, false,default,index,sizeof,true,typename,types},
  emphstyle       =[2]\color{yellow},
  emph            =[3]{types, parameters, gates, connections, submodules, allowunconnected},
  emphstyle       =[3]\color{orange},
  emph			  =[4]{int, string, double, @display, @class, volatile, xmldoc, xml, bool,const,this},
  emphstyle       =[4]\color{red}
}


%======================================================================
%	includeonly
%======================================================================

%\includeonly{		% Gibt an, welche Dateien der include-Befehl 
%			% tatsächlich einfuegen darf.
%  meta/metadaten		% Variablen setzen
%  ,titel		% Titelseite, Zusammenfassung und Inhaltsverzeichnis
% Alle per include einzulesenden Dateien müssen hier angegeben sein!
%  ,anleitung		% Anleitung zur Nutzung der Vorlage
%}

% Aufteilung der Arbeit in folgende Bestandteile in angeg. Reihenfolge
%
% Titelseite
% Bibliographische Beschreibung (Rückseite der Titelseite)
% (Aufgabenstellung)
% Danksagung
% Abstract
% Inhaltsverzeichnis
% Tabellenverzeichnis
% Abbildungsverzeichnis
% Einleitung
% Inhalt
% Zusammenfassung/Ausblick
% (Verzeichnis verwendeter Terme)
% Arbeitsaufteilung
% Glossar (Abkürzungsverzeichnis)
% (Index)
% Literaturverzeichnis
% (Thesen)
% (Selbstständigkeitserklärung)
% Anhänge