\chapter{Einleitung}

Die Verwendung von Sensoren steigt in der heutigen Zeit mehr und mehr an. Sensorknoten unterscheiden sich im Kern nicht von herkömmlichen Computern und sind zusätzlich mit Sensoren und oft mit Batterien und Funkmodulen ausgestattet. Da Computerbauteile bei gleicher Leistung kleiner und kleiner werden, ist es nicht verwunderlich dass auch Sensorknoten immer kleiner werden. Mit diesen kleinen Knoten ist es möglich ganze Netzwerke von Sensoren zu erschaffen, die miteinander kommunizieren. So können beispielsweise die Umgebungsparameter großer Naturflächen detailliert untersucht werden, ohne dass eine große Forschungsstation aufgebaut werden müsste. Stattdessen kann man viele kleine Sensorknoten in der Umwelt verteilen, die miteinander in Kontakt stehen. \newline
In der folgenden Arbeit soll ein solches Netz mit Hilfe von Omnet++\cite{omnet} simuliert werden.